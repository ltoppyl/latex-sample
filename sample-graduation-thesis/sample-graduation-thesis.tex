%--------------------

% モジュールのインポート

%--------------------
\documentclass[11pt,a4j]{jreport}
\usepackage{comment}
\usepackage{float}
\usepackage{color}
\usepackage{multicol}
\usepackage[dvipdfmx]{pict2e}
\usepackage{wrapfig}
\usepackage{graphicx}
\usepackage{bm}
\usepackage{url}
\usepackage{underscore}
\usepackage{colortbl}
\usepackage{tabularx}
\usepackage{fancyhdr}
\usepackage{ulem}
\usepackage{cite}
\usepackage{amsmath,amssymb,amsfonts}
\usepackage{mathtools}
\usepackage{algorithmic}
\usepackage{textcomp}
\usepackage{xcolor}
\usepackage{enumerate}
\usepackage{otf}
\usepackage{remreset}
\usepackage{here}
\usepackage[ipaex]{pxchfon}
\usepackage[top=30truemm,bottom=30truemm,left=25truemm,right=25truemm]{geometry}
\setcounter{secnumdepth}{4}

% 図表式番号を通し番号に設定
\makeatletter 
\@removefromreset{figure}{chapter}
\def\thefigure{\arabic{figure}}
\@removefromreset{table}{chapter}
\def\thetable{\arabic{table}}
\@removefromreset{equation}{chapter}
\def\theequation{\arabic{equation}}
\makeatother


\begin{document}
%--------------------

% 表紙

%--------------------
  \thispagestyle{empty} % このページの下にページ数を表示しない
    \begin{center}
      {\Huge \textgt{卒 業 論 文} \par} 
      {\LARGE {Graduation Thesis} \par}
      \vspace{20mm}
    \end{center}

    \begin{table}[htb]
      \begin{tabular}{ll}
        \vspace{2mm}
        {\LARGE \textgt {題 \hfill 目}} & 
        \begin{tabular}[t]{@{}l@{}}
          {\LARGE {タイトル1行目}}\\
          {\LARGE {タイトル2行目}}
        \end{tabular}\\
        \vspace{12mm}
        {\large {T \hfill I \hfill T \hfill L \hfill E:}} & 
        \begin{tabular}[t]{@{}l@{}}
          {\LARGE {Title First Line}}\\
          {\LARGE {Title Second Line}}\\
          {\LARGE {Title Third Line}}\\
        \end{tabular}\\
        \vspace{2mm}
        {\LARGE \textgt {指 \hfill 導 \hfill 教 \hfill 員}} & {\LARGE {◯◯ 教授}} \\
        \vspace{12mm}
        {\large {SUPERVISOR(S):}} & {\large {Prof. ◯◯}} \\
        \vspace{2mm}
        {\LARGE \textgt {提 \hfill 出 \hfill 者}} & {\LARGE {名前}} \\
        \vspace{12mm}
        {\large {A \hfill U \hfill T \hfill H \hfill O \hfill R:}} & {\large {NAME}} \\
        \vspace{2mm}
        {\LARGE \textgt {提 \hfill 出 \hfill 年 \hfill 月}} & {\LARGE {令和4年2月}} \\
        \vspace{12mm}
        {\large {D \hfill A \hfill T \hfill E:}} & {\large {February, 2022}}
      \end{tabular}
      \vspace{15mm}
    \end{table}

    \begin{center}
      {\LARGE \textgt{山口大学工学部知能情報工学科} \par} 
      {\large Department of Information Science \& Engineering, \par}
      {\large Faculty of Engineering, Yamaguchi University \par}
    \end{center}

  \newpage


  %--------------------

  % 概要

  %--------------------
  \thispagestyle{empty}
  \begin{center}
    \LARGE \textgt {概要}
  \end{center}
  概要
  \begin{center}
    \LARGE \textgt {Abstract}
  \end{center}
  Abstract

  \newpage


  %--------------------

  % 目次の表示

  %--------------------
  {
    \makeatletter
    \let\ps@jpl@in\ps@empty
    \makeatother
    \pagestyle{empty}
    \tableofcontents
    \newpage
  }

  
  \lhead{\rightmark}
  \renewcommand{\chaptermark}[1]{
    \markboth{第\ \normalfont\thechapter\ 章~~#1}{}
  }


  %--------------------

  % 本文

  %--------------------
  \setcounter{page}{1} %ここからページ数をカウントする


  %--------------------

  % はじめに

  %--------------------
  \chapter{はじめに}
    \section{研究の背景}
      研究の背景
    \section{研究の目的}
      研究の目的
    \section{本論文の構成}
      本論文の構成
  \newpage


  %--------------------

  % 関連研究

  %--------------------
  \chapter{関連研究}
    \section{関連研究1}
    \section{関連研究2}
    \section{関連研究3}
  \newpage


  %--------------------

  % 提案手法

  %--------------------
  \chapter{提案手法}
    \section{提案手法1}
    \section{提案手法2}
    \section{提案手法3}
  \newpage


  %--------------------

  % 実験

  %--------------------
  \chapter{実験}
    \section{使用データの概要}
    \section{実験条件}
    \section{実験結果}

  \newpage


  %--------------------

  % 考察

  %--------------------
  \chapter{考察}

  \newpage


  %--------------------

  % 謝辞

  %--------------------
  \chapter*{謝辞} %章を付けずにタイトル表示
  \addcontentsline{toc}{chapter}{謝辞} %章立てせずに目次に追加するおまじない
  謝謝

  \newpage


  %--------------------
  
  % 参考文献
  
  %--------------------
  \addcontentsline{toc}{chapter}{参考文献} %章立てせずに目次に追加するおまじない
  \renewcommand{\bibname}{参考文献} %これがないと,タイトルが「関連図書」になってしまう
  \begin{thebibliography}{99}
    \bibitem{References1}
      参考文献1

    \bibitem{References2}
      参考文献2
  \end{thebibliography}


  %--------------------
  
  % 付録
  
  %--------------------
  \chapter*{付録} %章を付けずにタイトル表示
  \addcontentsline{toc}{chapter}{付録} %章立てせずに目次に追加するおまじない
    \appendix
    \setcounter{section}{0} % section の番号をゼロにリセットする
    \renewcommand{\thesection}{\Alph{section}}
    \section{付録A}
      \subsection{付録A-1}

    \section{付録B}
      \subsection{付録B-1}

\end{document}
