%--------------------

% モジュールのインポート

%--------------------
\documentclass[a4paper,11pt]{jsarticle}
\usepackage{amsmath,amsfonts}
\usepackage{bm}
\usepackage[dvipdfmx]{graphicx}
\usepackage{titlesec}

% 参考文献のタイトルの表示の調整用
\titleformat{\section}[hang]
{\normalsize\gtfamily}
{\thesection}{.1em}{}[]
\titlespacing{\section}
{0pt}{.1ex plus .1ex minus .1ex}{0pt}


%--------------------

% タイトル

%--------------------
\makeatletter
\def\@maketitle{
\begin{center}
  {\textgt{レジュメのタイトル} \par} 
  {{学科 \quad 研究室 \quad 姓 \ 名} \par}
  {{指導教員: \quad 姓\ 名 \ 教授} \par}
  \vspace{3mm}
\end{center}
}


%--------------------

% 本文

%--------------------
\begin{document}
  \twocolumn % 書式を2段落に設定
  \maketitle % タイトル部分の表示

  \leftline{1. 図のサンプル}
    図 \ref{fig:sample-image1} のようにして, 図番号は参照できる. \\
    \begin{figure}[htbp]
      \centering
      \includegraphics[width=70mm]{sample-image1.png}
      \caption{図のサンプル1} % 表示される図のタイトル
      \label{fig:sample-image1} % 文章中で参照するためのラベル
    \end{figure}\\

    図番号は、自動で付与されるため, 図 \ref{fig:sample-image2} のようにすると, 違う図を参照できる. \\
    \begin{figure}[htbp]
      \centering
      \includegraphics[width=60mm]{sample-image2.png}
      \caption{図のサンプル2} 
      \label{fig:sample-image2} 
    \end{figure}\\

    \par
  
  \leftline{2. 数式のサンプル}
    文中で数式を利用するときは, $ f(x) = ax^{2} + bx + c $ のようにして書くと良い. \\
    数式に番号をつけて参照したい時は、図と同じように, (\ref{eq:sample-formula1})式 のようにすると良い.
    \begin{align}
      sum \ = \ \sum_{i=1}^{N} \ a_{i}
      \label{eq:sample-formula1}
    \end{align}\\

    複数行の数式で, 1つの番号を振りたい時は, 下のようにしてできる. (他にも方法はあるかも!?)\\
    \begin{equation}
      \begin{split}
        sum \ &= \ a_{1} + a_{2} + \cdots + a_{N} \\
        &= \sum_{i=1}^{N} \ a_{i}
        \label{eq:sample-formula2}
      \end{split}
    \end{equation}\\

    数式の細かい書き方に関しては, Google 先生で検索!! \\
    
    \par
  
    \leftline{3. 表のサンプル}
      表番号の参照も. 表 \ref{table:sample-table} のようにする. \\ 
      \begin{table}[htbp]
        \caption{表のサンプル} % 表のタイトル
        \label{table:sample-table}
        \centering
        \begin{tabular}{l|ll}
            & A  & B  \\ \hline
          A & 10 & 20 \\
          B & 30 & 40
        \end{tabular}
      \end{table}

      \par

    \leftline{4. 参考文献の参照}
      参考文献の参照は, \cite{References1} , \cite{References2} のようにして行う. \\

      \par

    %--------------------

    % 参考文献

    %-------------------- 
    \begin{thebibliography}{99}
      \bibitem{References1}
        参考文献1
      \bibitem{References2}
        参考文献2
        
    \end{thebibliography}
\end{document}
